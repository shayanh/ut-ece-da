\documentclass[11.5pt,a4paper,oneside]{article}
\usepackage{programming}
\usepackage{mathtools}
\DeclarePairedDelimiter{\floor}{\lfloor}{\rfloor}

\renewcommand{\contestname}{
دانشگاه تهران، دانشکده مهندسی برق و کامپیوتر \\
تحلیل و طراحی الگوریتم‌ها \\
}

\renewcommand{\contestauthor}{
تمرین کامپیوتری اول \\ موعد تحویل: دوشنبه ۶ اسفند ۹۷، ساعت ۲۳:۵۵ \\ طراح:‌ نام طراح،
\texttt{johndoe@example.com}
}

\begin{document}

{\noindent \Large \bf \contestname}
{\contestauthor}

\begin{flushleft}
\nothing\\[-3.2cm]
\includegraphics[height=2.5cm]{./ut-eng.png}
\end{flushleft}

\def\problemCode{DNA}
\def\problemEnglishTitle{Foreign DNA}
\def\problemFarsiTitle{دی‌ان‌ای فضایی}
\def\timeLimit{$2$ \second}
\def\memLimit{$256$ \megabytes}
\begin{problem}
موجودات فضایی‌ای که به‌تازگی کشف شده‌اند مشخص شده‌است که ویژگی عجیبی در DNA خود دارند. اولا که DNA شان فقط شامل حروف ABC می‌باشد، همچنین در هر زیر رشته‌ای که از ابتدای رشته شروع می‌شود تعداد A ها بیشتر مساوی تعداد B ها و همچنین تعداد B ها بیشتر مساوی تعداد C ها می‌باشد. در نهایت نیز تعداد A ها و B ها و C هایشان با هم برابر است. مثلا برای رشته‌هایی که دقیقا دو عدد A دارند ۵ حالت زیر وجود دارد.
\begin{shortitems}
\item
AABBCC
\item
AABCBC
\item
ABABCC
\item
ABACBC
\item
ABCABC
\end{shortitems}
از شما خواسته شده است که با دانستن تعداد A های یک رشته تعداد حالت‌های مختلف را برای آن رشته محاسبه کنید. مثلا برای ۲ پاسخ برابر ۵ است.

\inputDescription
در خط اول ورودی عدد T به شما داده می‌شود. در T خط بعدی تعداد A های هر تست کیس به شما داده می‌شود. این تعداد همواره مثبت و کمتر مساوی ۶۰ می‌باشد.

\outputDescription
به‌ازای هر تست کیس تعداد رشته‌های مختلف صحیح را چاپ کنید.

\sampleIO

\begin{example}
\exmp{%
2
1
3
}{%
1
42
}%
\end{example}

% \sampleIODescription
% اگر فرض کنیم
% $W=abcdefgh$
% ، داریم
% $w_q=w_4=d$
% و
% $\phi(W)=fehgbadc$
% . در رشته 
% $\phi(W)$
%  حرف 
% $w_q$
% ، ۸امین حرف است و بنابراین جواب مساله برابر ۸ خواهد بود.

\end{problem}


\def\problemCode{Realtime}
\def\problemEnglishTitle{Realtime Network}
\def\problemFarsiTitle{شبکه‌های بلادرنگ}
\def\timeLimit{$1$ \second}
\def\memLimit{$32$ \megabytes}
\begin{problem}
در سیستم‌های بلادرنگ قطعیت فاکتور مهمی است. به این دلیل گاهی طراحان این سیستم‌ها ترجیح می‌دهند که دقت و قطعیت را فدای عملکرد اتفاقی کنند. در این سوال قرار است مودم‌های شبکه‌ای به فروش برسند که در دسته‌ای از آنها حداکثر تاخیر در عملکردشان کمتر اکید از یک عدد می‌باشد و در دسته‌ای دیگر حداقل تاخیر در عملکردشان بیشتر اکید از یک عدد می‌باشد. به همین ترتیب دسته‌ای از مسئولان شبکه علاقه‌مند به خرید مودم‌های اکیدا سریع‌تر از یک آستانه و دسته‌ای دیگر علاقه‌مند به مودم‌های اکیدا کند‌تر از یک آستانه می‌باشند.

\inputDescription
در خط اول ورودی عدد $N$ قرار گرفته است که نشان دهنده‌ی تعداد مودم‌های بازار که برابر تعداد مسئولین شبکه نیز می‌باشد است. در $N$  خط بعدی ابتدا آستانه‌ی عملکرد مودم‌ها قرار گرفته‌اند. اگر این عدد مثبت باشد به این مفهوم است که این مودم حداکثر این مقدار تاخیر عملکرد خواهد داشت. در عین حال اگر منفی باشد یعنی حداقل قدر مطلق این میزان تاخیر دارد.
سپس در $N$ خط بعدی مشخصات مسئولین شبکه آمده است. اگر عدد مثبت باشد به این مفهوم است که او بدنبال مودمی است که حداقل تاخیرش کمتر از این میزان باشد. در عین حال اگر منفی باشد یعنی او بدنبال مودمی است که حداکثر تاخیرش بیشتر از قدر مطلق این عدد باشد.
به جز $N$ قدر مطلق تمامی اعداد بین ۱۵۰۰ تا ۲۵۰۰ می‌باشد. همچنین $N$ حداقل یک و حداکثر ۱۰۰۰۰۰ می‌باشد.

\outputDescription
در تنها خط خروجی حداکثر تعداد مودم‌هایی که می‌توانند فروخته شوند را بنویسید.

\constraints
\begin{shortitems}
\item $1 \le N \le 10^5$
% \item $0 \le A_{i, j} \le 9$
\end{shortitems}

\sampleIO

\begin{example}
\exmp{%
1
1600
-1600
}{%
0
}%
\exmp{%
1
1600
-1700
}{1}%
\exmp{%
2
-1900 -2300
2000 1800
}{2}%
\end{example}

% \sampleIODescription


\end{problem}

\def\problemCode{Compression}
\def\problemEnglishTitle{Image Compression}
\def\problemFarsiTitle{فشرده‌سازی تصویر}
\def\timeLimit{$2$ \second}
\def\memLimit{$256$ \megabytes}
\begin{problem}
یک تصویر سیاه و سفید از مجموعه‌ای از پیکسل‌ها تشکیل شده است که میزان تیرگی هرکدامشان یک عدد بین ۰ تا ۲۵۵ می‌باشد. در یک روش فشرده سازی ابتدا $k$ مقدار مبنا انتخاب می‌کنیم و سپس میزان تیرگی هر پیکسل را با تیرگی نزدیک‌ترین مبنا به آن جایگزین می‌کنیم. بدین ترتیب تنوع رنگی تصویرمان کاهش می‌یاد و فشرده‌تر می‌شود. میزان خطای این روش را نیز برابر مربع فاصله‌ی هر پیکسل از مقدار مبنایی که برای آن انتخاب شده در نظر می‌گیریم.
وظیفه‌ی شما محاسبه‌ی کمترین میزان خطای ممکن می‌باشد.

\inputDescription
در خط اول دو عدد $( 1 \leq d \leq 256)\:   d$ و $(1 \leq k \leq d)\:k$ به شما داده می‌شود. $d$ تنوع تیرگی‌های تصویر و $k$ تعداد نقاط مبنایی است که می‌خواهیم انتخاب کنیم. در $d$
 خط بعدی دو عدد $(0 \leq c \leq 255)\: c\:$ و $(1 \leq e \leq 2^{26})\: e \:$
به شما داده می‌شود که به ترتیب نشان‌گر میزان تیرگی و تعداد آن در کل تصویر می‌باشد. تضمین می‌شود که تیرگی‌ها به صورت صعودی می‌باشند. 

\outputDescription
کمینه‌ی خطای ممکن را در یک خط چاپ کنید.

\begin{example}
\exmp{%
2 1
25 30000
100 10000
}{42190000}%
\exmp{%
2 1
25 30000
100 10000}{0}%
\exmp{%
4 2
0 30000
50 30000
100 30000
255 30000}{150000000}%
\end{example}

\end{problem}
\end{document}
