\documentclass[11.5pt,a4paper,oneside]{article}
\usepackage{programming}
\usepackage{mathtools}
\DeclarePairedDelimiter{\floor}{\lfloor}{\rfloor}

\renewcommand{\contestname}{
دانشگاه تهران، دانشکده مهندسی برق و کامپیوتر \\
تحلیل و طراحی الگوریتم‌ها \\
}

\renewcommand{\contestauthor}{
تمرین کامپیوتری دوم \\ موعد تحویل: دوشنبه ۱۹ فروردین ۹۸، ساعت ۲۳:۵۵ \\ طراح:‌ آبتین باطنی،
\texttt{abtinbateni+da-ca@gmail.com}
}

\begin{document}

{\noindent \Large \bf \contestname}
{\contestauthor}

\begin{flushleft}
\nothing\\[-3.2cm]
\includegraphics[height=2.5cm]{../../../../static/pics/ut-eng.png}
\end{flushleft}

\def\problemCode{DNA}
\def\problemEnglishTitle{Foreign DNA}
\def\problemFarsiTitle{دی‌ان‌ای فضایی}
\def\timeLimit{$2$ \second}
\def\memLimit{$256$ \megabytes}
\begin{problem}
به‌تازگی موجودات فضایی‌ای کشف شده‌اند که ویژگی عجیبی در DNA خود دارند. اولا که DNA شان فقط شامل حروف ABC می‌باشد، همچنین در هر زیر رشته‌ای که از ابتدای رشته شروع می‌شود تعداد A ها بیشتر مساوی تعداد B ها و همچنین تعداد B ها بیشتر مساوی تعداد C ها می‌باشد. در نهایت نیز تعداد A ها و B ها و C هایشان با هم برابر است. مثلا برای رشته‌هایی که دقیقا دو عدد A دارند ۵ حالت زیر وجود دارد:
\begin{shortitems}
\item
AABBCC
\item
AABCBC
\item
ABABCC
\item
ABACBC
\item
ABCABC
\end{shortitems}
از شما خواسته شده است که با دانستن عدد
$N$
که برابر تعداد A های یک رشته است، تعداد حالت‌های مختلف را برای آن رشته محاسبه کنید. مثلا برای ۲ پاسخ برابر ۵ است.

\inputDescription
در خط اول ورودی عدد $T$ به شما داده می‌شود. در $T$ خط بعدی عدد $N$ برای هر تست کیس به شما داده می‌شود.

\outputDescription
به‌ازای هر تست کیس تعداد رشته‌های مختلف صحیح را چاپ کنید.

\constraints
\begin{shortitems}
	\item $1 \le T \le 65$
	\item $1 \le N \le 60$
\end{shortitems}

\sampleIO

\begin{example}
\exmp{%
2
1
3
}{%
1
42
}%
\end{example}

\end{problem}

\def\problemCode{Realtime}
\def\problemEnglishTitle{Realtime Network}
\def\problemFarsiTitle{شبکه‌های بلادرنگ}
\def\timeLimit{$1$ \second}
\def\memLimit{$32$ \megabytes}
\begin{problem}
در سیستم‌های بلادرنگ قطعیت فاکتور مهمی است. گاهی طراحان این سیستم‌ها ترجیح می‌دهند که عملکرد اتفاقی معمولا سریع را با دقت و قطعیت ولی اندکی کاهش کارآیی جایگزین کنند. در این سوال قرار است مودم‌های شبکه‌ای به فروش برسند که در دسته‌ای از آنها حداکثر تاخیر در عملکردشان کمتر اکید از یک عدد می‌باشد و در دسته‌ای دیگر حداقل تاخیر در عملکردشان بیشتر اکید از یک عدد می‌باشد. به همین ترتیب دسته‌ای از مسئولان شبکه علاقه‌مند به خرید مودم‌های اکیدا سریع‌تر از یک آستانه و دسته‌ای دیگر علاقه‌مند به مودم‌های اکیدا کند‌تر از یک آستانه می‌باشند.

هر مسئول شبکه حداکثر یک مودم با شرایط مورد نظرش را خریداری خواهد کرد و همچنین تعداد مودم‌های موجود در بازار برابر تعداد مسئولین شبکه است. از شما خواسته شده است حداکثر تعداد مودم‌هایی که می‌توانند به فروش برسند را به‌ دست آورید.

\inputDescription
در خط اول ورودی عدد $N$ برابر با تعداد مودم‌های موجود در بازار، آمده است. در $N$ خط بعدی آستانه‌ی عملکرد مودم‌ها، $A_i$، آمده است. اگر $A_i$ مثبت باشد به این معنی است که این مودم حداکثر $A_i$ میلی‌ثانیه تاخیر دارد و اگر منفی باشد یعنی حداقل به مقدار $|A_i|$ میلی‌ثانیه تاخیر دارد.
سپس در $N$ خط بعدی مشخصات مسئولین شبکه، $B_i$، آمده است. اگر $B_i$ مثبت باشد به این معنی است که مسئول شبکه $i$ام به دنبال مودمی است که حداقل تاخیرش کمتر از $B_i$ میلی‌ثانیه باشد و اگر منفی باشد یعنی او به دنبال مودمی است که حداکثر تاخیرش بیشتر از $|B_i|$ میلی‌ثانیه باشد.

\outputDescription
در تنها خط خروجی حداکثر تعداد مودم‌هایی که می‌توانند فروخته شوند را بنویسید.

\constraints
\begin{shortitems}
\item $1 \le N \le 10^5$
\item $1500 \le A_i, B_i \le 2500$
\end{shortitems}

\sampleIO

\begin{example}
\exmp{%
1
1600
-1600
}{%
0
}%
\exmp{%
1
1600
-1700
}{1}%
\exmp{%
2
-1900 -2300
2000 1800
}{2}%
\end{example}

% \sampleIODescription


\end{problem}

\def\problemCode{Compression}
\def\problemEnglishTitle{Image Compression}
\def\problemFarsiTitle{فشرده‌سازی تصویر}
\def\timeLimit{$2$ \second}
\def\memLimit{$256$ \megabytes}
\begin{problem}
یک تصویر سیاه و سفید از مجموعه‌ای از پیکسل‌ها تشکیل شده است که میزان تیرگی هرکدامشان یک عدد بین ۰ تا ۲۵۵ می‌باشد. در یک روش فشرده سازی ابتدا $k$ عدد
$v_1, v_2, ..., v_k$
را به عنوان مبنا انتخاب می‌کنیم و سپس میزان تیرگی هر پیکسل را با تیرگی نزدیک‌ترین مبنا به آن جایگزین می‌کنیم. بدین ترتیب تنوع رنگی تصویرمان کاهش می‌یاد و فشرده‌تر می‌شود. 

با این فرض که تیرگی اولیه پیکسل‌های عکس
$r_1, r_2, ..., r_n$
باشد و مبنای انتخاب شده $v_1, v_2, ..., v_k$ باشد، میزان خطا به این صورت محاسبه می‌شود:
$$ \sum _{i=1}^ n \min _{1\leq j \leq k} (r_ i - v_ j)^2$$
از شما خواسته شده برای عکس ورودی، با انتخاب بهترین مبنا، مقدار کمینه خطا را بیابید.

\inputDescription
در خط اول دو عدد $d$ و $k$ به شما داده می‌شود. $d$ تنوع تیرگی‌های تصویر و $k$ تعداد نقاط مبنایی است که می‌خواهیم انتخاب کنیم. در $d$
خط بعدی دو عدد $c_i$ و $e_i$
به شما داده می‌شود که به ترتیب نشان‌گر میزان تیرگی و تعداد آن در کل تصویر می‌باشد. تضمین می‌شود که تیرگی‌ها به صورت صعودی می‌باشند. 

\outputDescription
کمینه‌ی خطای ممکن را در یک خط چاپ کنید.

\constraints
\begin{shortitems}
	\item $1 \le d \le 255$
	\item $1 \le k \le d$
	\item $0 \le c_i \le 255$
	\item $1 \le e_i \le 2^{26}$
\end{shortitems}

\begin{example}
\exmp{%
2 1
25 30000
100 10000
}{42190000}%
\exmp{%
2 1
25 30000
100 10000}{0}%
\exmp{%
4 2
0 30000
50 30000
100 30000
255 30000}{150000000}%
\end{example}

\end{problem}
\end{document}
