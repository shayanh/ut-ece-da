\documentclass[11.5pt,a4paper,oneside]{article}
\usepackage{programming}
\usepackage{mathtools}
\DeclarePairedDelimiter{\floor}{\lfloor}{\rfloor}

\renewcommand{\contestname}{
دانشگاه تهران، دانشکده مهندسی برق و کامپیوتر \\
تحلیل و طراحی الگوریتم‌ها \\
}

\renewcommand{\contestauthor}{
تمرین کامپیوتری دوم \\ موعد تحویل: دوشنبه ۱۹ فروردین ۹۸، ساعت ۲۳:۵۵ \\ طراح:‌ آبتین باطنی،
\texttt{abtinbateni+da-ca@gmail.com}
}

\begin{document}

{\noindent \Large \bf \contestname}
{\contestauthor}

\begin{flushleft}
\nothing\\[-3.2cm]
\includegraphics[height=2.5cm]{../../../../static/pics/ut-eng.png}
\end{flushleft}

\def\problemCode{DNA}
\def\problemEnglishTitle{Space DNA}
\def\problemFarsiTitle{دی‌ان‌ای فضایی}
\def\timeLimit{$2$ \second}
\def\memLimit{$256$ \megabytes}
\begin{problem}
به‌تازگی موجودات فضایی‌ای کشف شده‌اند که ویژگی عجیبی در DNA خود دارند. اولا که DNA شان فقط شامل حروف ABC می‌باشد، همچنین در هر زیر رشته‌ای که از ابتدای رشته شروع می‌شود تعداد A ها بیشتر مساوی تعداد B ها و همچنین تعداد B ها بیشتر مساوی تعداد C ها می‌باشد. در نهایت نیز تعداد A ها و B ها و C هایشان با هم برابر است. مثلا برای رشته‌هایی که دقیقا دو عدد A دارند ۵ حالت زیر وجود دارد:
\begin{shortitems}
\item
AABBCC
\item
AABCBC
\item
ABABCC
\item
ABACBC
\item
ABCABC
\end{shortitems}
از شما خواسته شده است که با دانستن عدد
$N$
که برابر تعداد A های یک رشته است، تعداد حالت‌های مختلف را برای آن رشته محاسبه کنید. مثلا برای ۲ پاسخ برابر ۵ است.

\inputDescription
در خط اول ورودی عدد $T$ به شما داده می‌شود. در $T$ خط بعدی عدد $N$ برای هر تست کیس به شما داده می‌شود.

\outputDescription
به‌ازای هر تست کیس تعداد رشته‌های مختلف صحیح را چاپ کنید.

\constraints
\begin{shortitems}
	\item $1 \le T \le 65$
	\item $1 \le N \le 60$
\end{shortitems}

\sampleIO

\begin{example}
\exmp{%
2
1
3
}{%
1
42
}%
\end{example}

\end{problem}

\def\problemCode{Realtime}
\def\problemEnglishTitle{Realtime Teammate}
\def\problemFarsiTitle{همگروهی بلادرنگ}
\def\timeLimit{$1$ \second}
\def\memLimit{$32$ \megabytes}
\begin{problem}
در یک کلاس طراحی الگوریتم تمرین‌ها به صورت گروهی انجام می‌شوند. از آنجایی که فرآیند تیم شدن همواره بر عهده‌ی خود دانشجو‌ها گذاشته میشد اینبار استاد تصمیم به دخالت گرفته است. در تصمیمی که بلادرنگ گرفته شد نمره‌های درس ساختمان‌داده‌ی دانشجویان ملاک قرار داده شد و سپس به طور اتفاقی دانشجویان به دو گروه مساوی تقسیم شدند. هر دانشجو موظف شد که با یک فرد از گروه مقابل همگروه شود و همچنین هر فرد به طور مجزا تصمیم گرفته که با یک فرد از خود قوی‌تر و یا ضعیف‌تر همگروه شود (هیچ‌کس با هم سطحش همگروه نمی‌شود). با وجود این شرایط حداکثر چند تیم می‌تواند به وجود بیاید؟

\inputDescription
در خط اول ورودی عدد $N$ (تعداد اعضای هر گروه) به شما داده می‌شود. 
در $N$ خط بعدی $A_i$ها (نمره‌های دانشجویان) به شما داده می‌شود. اگر این عدد مثبت باشد به این معنی است که او تمایل به تیم شدن با فرد قوی‌تر از خودش و در غیر این صورت تمایل به تیم شدن با فرد ضعیف‌تر از خودش را دارد. در $N$ خط بعدی  $B_i$ها (نمره‌های گروه دوم) به شما داده شده که همانند اعضای گروه اول در صورتی که نمره‌ی کسی مثبت باشد او تمایل به تیم شدن با فرد قوی‌تر از خودش و در غیر این صورت تمایل تیم شدن با فرد ضعیف‌تر از خودش دارد.

\outputDescription
در تنها خط خروجی حداکثر تعداد تیم‌هایی که می‌توانند تشکیل شوند را چاپ کنید.
\constraints
\begin{shortitems}
\item $1 \le N \le 10^5$
\item $1500 \le |A_i|, |B_i| \le 2500$
\end{shortitems}

\sampleIO

\begin{example}
\exmp{%
1
1600
-1600
}{%
0
}%
\exmp{%
1
1600
-1700
}{1}%
\exmp{%
2
-1900 -2300
2000 1800
}{2}%
\end{example}

% \sampleIODescription


\end{problem}

\def\problemCode{Compression}
\def\problemEnglishTitle{Image Compression}
\def\problemFarsiTitle{فشرده‌سازی تصویر}
\def\timeLimit{$2$ \second}
\def\memLimit{$256$ \megabytes}
\begin{problem}
یک تصویر سیاه و سفید از مجموعه‌ای از پیکسل‌ها تشکیل شده است که میزان تیرگی هرکدامشان یک عدد بین ۰ تا ۲۵۵ می‌باشد. در یک روش فشرده سازی ابتدا $k$ عدد
$v_1, v_2, ..., v_k$
را به عنوان مبنا انتخاب می‌کنیم و سپس میزان تیرگی هر پیکسل را با تیرگی نزدیک‌ترین مبنا به آن جایگزین می‌کنیم. بدین ترتیب تنوع رنگی تصویرمان کاهش می‌یاد و فشرده‌تر می‌شود. 

با این فرض که تیرگی اولیه پیکسل‌های عکس
$r_1, r_2, ..., r_n$
باشد و مبنای انتخاب شده $v_1, v_2, ..., v_k$ باشد، میزان خطا به این صورت محاسبه می‌شود:
$$ \sum _{i=1}^ n \min _{1\leq j \leq k} (r_ i - v_ j)^2$$
از شما خواسته شده برای عکس ورودی، با انتخاب بهترین مبنا، مقدار کمینه خطا را بیابید.

\inputDescription
در خط اول دو عدد $d$ و $k$ به شما داده می‌شود. $d$ تنوع تیرگی‌های تصویر و $k$ تعداد نقاط مبنایی است که می‌خواهیم انتخاب کنیم. در $d$
خط بعدی دو عدد $c_i$ و $e_i$
به شما داده می‌شود که به ترتیب نشانگر میزان تیرگی و تعداد آن در کل تصویر می‌باشد. تضمین می‌شود که تیرگی‌ها به صورت صعودی می‌باشند. 

\outputDescription
کمینه‌ی خطای ممکن را در یک خط چاپ کنید.

\constraints
\begin{shortitems}
	\item $1 \le d \le 255$
	\item $1 \le k \le d$
	\item $0 \le c_i \le 255$
	\item $1 \le e_i \le 2^{26}$
\end{shortitems}

\begin{example}
\exmp{%
2 1
25 30000
100 10000
}{42190000}%
\exmp{%
2 2
25 30000
100 10000}{0}%
\exmp{%
4 2
0 30000
50 30000
100 30000
255 30000}{150000000}%
\end{example}

\end{problem}
\end{document}
