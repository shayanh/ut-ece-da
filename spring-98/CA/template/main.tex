\documentclass[11.5pt,a4paper,oneside]{article}
\usepackage{programming}
\usepackage{mathtools}
\DeclarePairedDelimiter{\floor}{\lfloor}{\rfloor}

\renewcommand{\contestname}{
دانشگاه تهران، دانشکده مهندسی برق و کامپیوتر \\
تحلیل و طراحی الگوریتم‌ها \\
}

\renewcommand{\contestauthor}{
تمرین کامپیوتری اول \\ موعد تحویل: دوشنبه ۶ اسفند ۹۷، ساعت ۲۳:۵۵ \\ طراح:‌ نام طراح،
\texttt{johndoe@example.com}
}

\begin{document}

{\noindent \Large \bf \contestname}
{\contestauthor}

\begin{flushleft}
\nothing\\[-3.2cm]
\includegraphics[height=2.5cm]{../../../static/pics/ut-eng.png}
\end{flushleft}

\def\problemCode{encoding}
\def\problemEnglishTitle{}
\def\problemFarsiTitle{رمزنگاری}
\def\timeLimit{$2$ \second}
\def\memLimit{$256$ \megabytes}
\begin{problem}
می‌خواهیم رشته‌ای از حروف را با استفاده از تابع $\phi$ رمز کنیم.
برای رشته $W$، تابع
$\phi(W)$
به صورت زیر تعریف می‌شود:
\begin{shortitems}
\item
اگر
$|W|=1$
باشد، آنگاه
$\phi(W)=W$

\item
در غیر این صورت
$W=w_1 w_2 ... w_n$
و
$k=\floor*{n\over2}$
، آنگاه
$\phi(W) = \phi(w_n w_{n-1} ... w_{k+1}) + \phi(w_k w_{k-1} ... w_1)$
\end{shortitems}


برای مثال:
\begin{enumerate}
\item
$\phi(Ok)=\phi(k)+\phi(O)=k+O=kO$
\item
$\phi(abcd)=\phi(dc)+\phi(ba)=\phi(c)+\phi(d)+\phi(a)+\phi(b)=cdab$
\end{enumerate}

شما باید اندیس کاراکتر $w_q$ را در رشته
$\phi(W)$
پیدا کنید.

\inputDescription
در تنها خط ورودی عددهای $n$ و $q$ به ترتیب آمده‌اند. ($n$ نشان‌دهنده طول رشته $W$ است)

\outputDescription
عدد خواسته‌شده را در خروجی نمایش دهید.

\sampleIO

\begin{example}
\exmp{%
9 4
}{%
8
}%
\end{example}

\sampleIODescription
اگر فرض کنیم
$W=abcdefgh$
، داریم
$w_q=w_4=d$
و
$\phi(W)=fehgbadc$
. در رشته 
$\phi(W)$
 حرف 
$w_q$
، ۸امین حرف است و بنابراین جواب مساله برابر ۸ خواهد بود.

\end{problem}


\def\problemCode{school}
\def\problemEnglishTitle{}
\def\problemFarsiTitle{مدرسه}
\def\timeLimit{$2$ \second}
\def\memLimit{$256$ \megabytes}
\begin{problem}
در حیاط مدرسه «بوی‌ ماه مهر»، 
$n^2$
دانش‌آموز آن در یک مربع 
$n \times n$
ایستاده‌اند. قد دانش‌آموزی که در ردیف $i$ام و ستون $j$ام ایستاده‌است برابر
$A_{i, j}$
است. حال بعد از به صدا درآمدن زنگ، هر دانش‌آموز به چهار دانش‌آموز بالا، پایین، چپ و راست خود (در صورت وجود) نگاه می‌کند و به ناظم مدرسه می‌گوید که چند تا از آن‌ها از او قدشان بلندتر است. فرض می‌کنیم که دانش‌آموز در ردیف $i$ و ستون $j$ام، عدد
$B_{i, j}$
را گزارش کرده‌است.

به شما جدول $B$ داده‌شده‌ است. شما باید جدول $A$ که $B$ از روی آن بدست آمده است را پیدا کنید.

\inputDescription
در اولین عدد $n$ آمده‌است.
در هر کدام از $n$ خط بعدی، $n$ عدد می‌آیند که ماتریس $B$ را توصیف می‌کنند.
\outputDescription
اگر ماتریس $A$ متناظر با $B$ وجود داشت در $n$ خط خروجی درایه‌های آن را چاپ کنید و در غیر این صورت در تنها خط خروجی عبارت \LR{NO SOLUTION} را بنویسید.

\constraints
\begin{shortitems}
\item $1 \le n \le 3$
\item $0 \le A_{i, j} \le 9$
\end{shortitems}

\sampleIO

\begin{example}
\exmp{%
3
1 2 1
1 2 1
1 1 0
}{%
1 2 3
1 4 5
1 6 7
}%
\end{example}

\sampleIODescription
برای مثال
$B_{2, 2}=2$
است چرا که اگر مقدار
$A_{2, 2}$
را با همسایه‌های آن (یعنی 
$A_{1, 2}$، $A_{2, 1}$، $A_{2, 3}$ و $A_{3, 2}$)
مقایسه کنیم تنها ۲ عدد بزرگ‌تر از
$A_{2, 2}$
در بین‌شان هست.

\end{problem}

\end{document}