\documentclass[11.5pt,a4paper,oneside]{article}
\usepackage{programming}
\usepackage{mathtools}
\DeclarePairedDelimiter{\floor}{\lfloor}{\rfloor}

\renewcommand{\contestname}{
دانشگاه تهران، دانشکده مهندسی برق و کامپیوتر \\
تحلیل و طراحی الگوریتم‌ها \\
}

\renewcommand{\contestauthor}{
تمرین کامپیوتری اول \\ موعد تحویل: چهارشنبه ۲۲ اسفند ۹۷، ساعت ۲۳:۵۵ \\ طراح:‌ بهار باطنی،
\texttt{iambb5445@yahoo.com}
}

\begin{document}

{\noindent \Large \bf \contestname}
{\contestauthor}

\begin{flushleft}
\nothing\\[-3.2cm]
\includegraphics[height=2.5cm]{../../../../static/pics/ut-eng.png}
\end{flushleft}

\def\problemCode{painting wall}
\def\problemEnglishTitle{}
\def\problemFarsiTitle{رنگ زدن دیوار}
\def\timeLimit{$2$ \second}
\def\memLimit{$256$ \megabytes}
\begin{problem}

آقای خسته، یک اتاق با یک دیوار به شکل یک جدول 
$n \times n$
دارد که طبق مد روز، روی هر خانه ی قطر اصلی آن یک قاب عکس قرار گذاشته است.
(خانه های قطر اصلی، خانه هایی هستند که در مختصاتی به شکل
$(i, i)$
قرار دارند، یعنی شماره سطر و ستون برابر دارند.) 
او می خواهد دیوار را رنگ کند تا دلش باز شود اما از آنجا که خیلی خسته است و اصلا خسیس نیست، تصمیم گرفته برای این کار یک ربات نقاش بخرد.
این ربات می تواند از دستوری به شکل زیر برای رنگ آمیزی استفاده کند:

هر بار آقای خسته، ابتدا یک عدد
$k$
و سپس
$k$
شماره ستون
$c_1, c_2, ... ,c_k$
به آن می دهد
(
$1 \leq c_i \leq m$
)
تا ربات تمام این ستون ها را رنگ کند.

اما این ربات یک باگ کوچک دارد، در هر سطر اگر یک مانع برای رنگ کردن داشته باشد کل آن سطر را رنگ نمی کند. یعنی:
 
به ازای هر سطر 
$i$
(
$1 \leq i \leq n$
)،
ربات تقاطع این سطر و ستون های انتخاب شده
 را (که جمعا
$k$
خانه هستند)
برای رنگ زدن در نظر می گیرد.

\begin{shortitems}
\item

 در صورتی که حتی در یکی از این خانه ها قاب عکس بود، ربات بیخیال می شود و به سطر بعدی می رود.

\item

 در غیر اینصورت، ربات کل این 
 $k$
 خانه را رنگ آمیزی می کند.

\end{shortitems}

از آنجا که آقای خسته، خسته تر از این است که خودش فکر کند، و به علاوه دوست دارد دیوارش زود و با تعداد کمی دستور رنگ شود، از شما خواسته برنامه ای بنویسید که با داشتن
 $n$
 بتواند حداکثر 20 دستور بدهد که اگر ربات آن ها را طبق منطق خود اجرا کند، کل دیوار به جز محل قاب عکس ها رنگ شود.

\inputDescription

در تنها خط ورودی عدد
$n$
آمده است که تعداد سطر ها و ستون های دیوار است.
\outputDescription

ابتدا در یک خط
$m$
را چاپ کنید که تعداد دستورات پیشنهادی شما به ربات است.
سپس در
$m$
خط بعدی، در هر خط ابتدا
$k$
را چاپ کنید که تعداد ستون های انتخابی شما برای آن دستور است.
سپس
$k$
عدد چاپ کنید که نشان دهنده ی اندیس های ستون های انتخابی هستند.

\constraints
\begin{shortitems}
\item $1 \leq n \leq 1000$
\item $0 \leq m \leq 20$
\item $1 \leq k \leq n$
\end{shortitems}

\sampleIO

\begin{example}
\exmp{%
2
}{%
2
1 1
1 2
}%
\exmp{%
5
}{%
7
2 1 3
1 2
5 1 2 3 4 5
4 1 2 3 4
2 1 5
2 5 4
3 2 3 4
}%
\end{example}

\sampleIODescription
در مثال اول، در دستور اول خانه ی
$(2, 1)$
و در دستور دوم خانه ی
$(1, 2)$
رنگ می شود.

در مثال دوم در هفت دستور داده شده، خانه ها به شکل زیر پر می شوند:
\\
دستور اول: خانه های
$(2, 1), (2, 3), (4, 1), (4, 3), (5, 1), (5, 3)$
\\
دستور دوم: خانه های
$(1, 2), (3, 2), (4, 2), (5, 2)$
\\
دستور سوم: هیچ خانه ای رنگ نمی شود، چون هیچ سطری نداریم که هیچ خانه ی قاب عکس داری از آن انتخاب نشده باشد.
\\
دستور چهارم: خانه های
$(5, 1), (5, 2), (5, 3), (5, 4)$
که سه خانه ی اول قبلا رنگ شده بودند و تغییری در وضعیت آن ها حاصل نمی شود.
\\
دستور پنجم: خانه های
$(2, 1), (2, 5), (3, 1), (3, 5), (4, 1), (4, 5)$
که برخی قبلا رنگ شده بودند.
\\
دستور ششم: خانه های
$(1, 4), (1, 5), (2, 4), (2, 5), (3, 4), (3, 5)$
\\
دستور هفتم: خانه های
$(1, 2), (1, 3), (1, 4), (5, ,2), (5, 3), (5, 4)$
\\
در نهایت کل دیوار جز قاب عکس ها رنگ شده و در نتیجه صرف نظر از تعداد دستورات، پاسخ صحیح است. (البته این تعداد باید از 20 کمتر یا مساوی باشد.)
\end{problem}


\def\problemCode{managing restaurant}
\def\problemEnglishTitle{}
\def\problemFarsiTitle{رستوران داری}
\def\timeLimit{$2$ \second}
\def\memLimit{$256$ \megabytes}
\begin{problem}
جناب خسیس به تازگی یک رستوران تاسیس کرده است که در طبقه ی آخر ساختمانی ضد زلزله قرار دارد. این ساختمان به این صورت است که در صورت وقوع زلزله، فقط به چپ متمایل می شود ولی نمی ریزد. جناب خسیس همچنین
$n$
میز تهیه کرده و در رستوران گذاشته است.
میز
$i$
ام در فاصله ی
$d_i$
متر از درب ورود قرار دارد. (که می تواند مقداری منفی یا مثبت باشد؛ منفی یعنی سمت چپ درب قرار دارد و مثبت یعنی سمت راست آن است.)

مدیر ساختمان به جناب خسیس هشدار داده که اگر ساختمان به چپ متمایل شود، میز ها هم به چپ لیز می خورند مگر اینکه هر روز آن ها را بچسباند.
در واقع هر میز اگر چسبیده باشد حرکت نمی کند، وگرنه انقدر به چپ لیز می خورد تا به میز چسبیده ی سمت چپی اش برسد. (فرض کنید عرض میز ها انقدر کم است که چندین میز می توانند در یک فاصله از درب قرار گیرند.)
همچنین اگر میزی که نچسبیده، سمت چپش میز چسبیده ای نداشته باشد، از پنجره به بیرون پرت می شود و رستوران کلا بسته می شود.
اما جناب خسیس که خسیس است ولی اصلا خسته نیست، تصمیم می گیرد فقط بعضی میز ها را بچسباند و باقی میز ها را بعد از پایان زلزله، هل دهد و به جای اصلیشان برگرداند.

از آنجا که جناب خسیس بعضی روز ها سر کار نمی رود، مجبور است در صورت وقوع زلزله در آن روز ها، به شاگردش پول دهد تا میز ها را هل دهد. شاگرد او به ازای هر یک متری که میز هل دهد، هزار تومان می گیرد.
به علاوه پول چسباندن میز
$i$
ام،
$t_i$
هزار تومان است.
(که باز هم می تواند مثبت یا منفی باشد، بعضی وقت ها مدیر ساختمان به خاطر عوامل خیر خواهانه بعضی میز ها را مجانی چسب می زند و حتی به جناب خسیس پول می دهد.)

شما که دلتان به حال کسب و کار آقای خسیس و ضرر او هنگام زلزله سوخته، تصمیم گرفتید برنامه ای بنویسید که به او بگوید روز هایی که سر کار نمی رود، کدام میز ها را بچسباند تا در صورت وقوع زلزله، سرجمع پول چسباندن و هل دادن میز ها کمینه شود.

البته حواستان باشد میزی از پنجره بیرون نیفتد وگرنه رستوران بسته می شود.

\inputDescription
در اولین خط عدد
$n$
یعنی تعداد میز ها آمده است.

در خط دوم،
$n$
عدد می آید که
$d_i$ ها هستند.
همچنین فرض کنید هیچ دو میزی در ابتدا در یک نقطه قرار ندارند.
$(d_i \neq d_j)$

در نهایت در خط سوم،
$n$
عدد دیگر می آید که
$t_i$
ها هستند.

\outputDescription
خروجی تنها یک خط دارد که کمترین هزینه ی ممکن برای چسباندن و هل دادن میز ها بر حسب هزار تومان است.

\constraints
\begin{shortitems}
\item $1 \le n \leq 2800$
\item $-2^{30} \leq d_i, t_i \leq 2^{30}$
\end{shortitems}

\sampleIO

\begin{example}
\exmp{%
3
0 2 10
5 6 13
}{%
17
}%
\exmp{%
4
{-}4 {-}3 14 {-}1
100 {-}4 1 0
}{%
97
}%
\exmp{%
4
6 2 5 3
1 7 100 2
}{%
12
}%
\exmp{%
5
1 2 3 4 5
3 3 3 3 3
}{%
10
}%
\end{example}

\sampleIODescription
در مثال اول، کافی است تنها چپ ترین میز چسبانده شود. در اینصورت
$5$
هزار تومان هزینه ی چسبیدن آن است و در صورت وقوع زلزله، دو میز راستی به آن می چسبند که هزینه ی هل دادن آن ها، 
$2 + 10 = 12$ 
هزار تومان است. که جمعا
$17$
هزار تومان خرج می شود.

در مثال دوم، میز اول باید با وجود هزینه ی زیادی که دارد حتما چسبانده شود. وگرنه از پنجره بیرون می افتد.
میز دوم نیز با وجود اینکه فاصله ی چندانی با میز چپی ندارد، اگر چسبانده شود هزینه ها
$4$
هزار تومان کاهش میابد.
به همین ترتیب راه حل خوب این است که دو میز دیگر هم چسبانده شود و در مجموع
$97$
هزار تومان خرج می شود.

در مثال سوم، میز دوم حتما باید چسبانده شود. تا از پنجره بیرون نیفتد. اگر میز اول و دوم و چهارم را بچسبانیم کمترین هزینه را می دهیم که
$10$
هزار تومان هزینه ی چسباندن و
$2$
هزار تومان هزینه ی هل دادن است.

در مثال چهارم، بهترین حالت چسباندن میز  اول و چهارم است که باعث می شود سه میز دیگر لیز بخورند و هزینه ی هل دادن،
$4$
هزار تومان شود و هزینه ی کل،
$10$
هزار تومان.

\end{problem}

\end{document}