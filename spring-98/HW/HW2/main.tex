\documentclass{article}
\usepackage{tikz}
\usetikzlibrary{decorations.pathreplacing}
\usepackage{fancyhdr}
\usepackage{fullpage}
%\usepackage[margin=0.5in]{geometry}
\usepackage{amsmath}
%\usepackage[margin=0.5in]{geometry}
\usepackage{xepersian}
%\settextfont[BoldFont=Bold.ttf]{Regular.ttf}
%\usepackage{enumerate}
\date{}
\usepackage{xepersian}
\settextfont{XBNiloofar.ttf}
%\usepackage{enumerate}
\date{}

%\pagestyle{fancy}
%\fancyhead[RE,RO]{راست}
%\fancyhead[LE,LO]{چپ}
%\fancyhead[CE,CO]{وسط}

\def\vstrut#1{\rule{0pt}{#1}}
\def\nothing{\vstrut{0pt}}

\begin{document}


\begin{center}
{به نام خدا}
\end{center}

\hspace{8mm}

{\noindent \Large \textbf {
دانشگاه تهران، دانشکده مهندسی برق و کامپیوتر \\
تحلیل و طراحی الگوریتم‌ها \\
}}

{\noindent {
تمرین کتبی دوم \\
موعد تحویل: شنبه ۴ اسفند ۹۷، ساعت ۹:۰۰ \\
طراح: آبتین باطنی
\texttt{abtinbateni+da-hw@gmail.com}
}}

\begin{flushleft}
\nothing \\[-3.2cm]
\includegraphics[height=2.5cm]{./ut-eng.png}
\end{flushleft}

%\maketitle
\rule{\textwidth}{1pt}


\begin{enumerate}
\item
یک الگوریتم از $O(n^2)$ برای پیدا کردن بزرگ‌ترین زیردنباله‌ی غیر نزولی از دنباله اعداد
$X = {x_1, x_2, ..., x_n}$
ارائه دهید.
(ارائه الگوریتم از $O(nlogn)$ نمره اضافی دارد.)
\item
فرض کنید u و v دو رشته باشند. ما می‌خواهیم رشته u را به رشته‌ی v با عمل‌های زیر تبدیل کنیم:
\begin{itemize}
    \item حذف یک کارکتر
    \item اضافه کردن یک کاراکتر در یک مکان
    \item
    عوض کردن یک کارکتر
\end{itemize}
اگر طول دو رشته به ترتیب n و m باشد یک الگورتیم از O(nm) ارائه دهید که کمترین تعداد عملیات مورد نیاز را بشمارد.
\item
در رودخانه n نقطه وجود دارد که در آنها می‌توان قایق اجاره کرد. فرض کنید این نقاط در راستای رود‌خانه به ترتیب از ۱ تا n شماره‌گذاری شده‌اند. همچنین فرض کنید هزینه اجاره کردن قایق از نقطه i و رفتن تا خانه j برابر $a_{ij}$
باشد. روشی ارائه دهید که با کمترین هزینه از نقطه ۱ با اجاره کردن تعدادی قایق به نقطه‌ی n برسیم.
\begin{itemize}
    \item الگوریتمی از $O(n^2)$ ارائه دهید.
    \item با استفاده از روش Trick Hull Convex در چه صورتی می‌توان الگوریتم را بهینه‌تر کرد؟با اضافه کردن این شرط و با استفاده از این ایده الگوریتمی از O(n) برای حل مسئله ارائه دهید. (این بخش به طور کامل امتیازی است و این روش را در اینترنت جستجو کنید.)
\end{itemize}
\item
فرض کنید n شئ در اختیار داریم که می‌توانیم به ترتیبی آنها را پشت سر هم قرار داده و بین آنها علائم < و = قرار دهیم. تعداد راه‌های انجام این کار را بدست آورید. مثلا برای ،n=۲ ۳ روش و برای ،n=۳ ۱۳ روش وجود دارد. (فرض کنید فاکتوریل هر عددی را در O(1) می‌توان حساب کرد.)
\item
الگوریتمی از O(nk) ارائه دهید 
که تعداد راه‌های ساخت عدد n را با استفاده از سکه‌های $c_1, ..., c_k$ تومانی را با استفاده از O(n) خانه‌ی حافظه بشمارد.
\item
در یک کارخانه چوب‌بری عجیب برای اینکه یک تکه چوب را در یک مرحله به k تکه برش بزنند $c_k$ تومن پول گرفته می‌شود. الگوریتمی ارائه دهید که یک تکه چوب را با کمترین خرج به n تکه تقسیم کند.
\end{itemize}

\end{enumerate}

\end{document}
