\documentclass{article}
\usepackage{tikz}
\usetikzlibrary{decorations.pathreplacing}
\usepackage{fancyhdr}
\usepackage{fullpage}
%\usepackage[margin=0.5in]{geometry}
\usepackage{xepersian}
\settextfont{XBNiloofar.ttf}
%\usepackage{enumerate}
\date{}

%\pagestyle{fancy}
%\fancyhead[RE,RO]{راست}
%\fancyhead[LE,LO]{چپ}
%\fancyhead[CE,CO]{وسط}

\def\vstrut#1{\rule{0pt}{#1}}
\def\nothing{\vstrut{0pt}}

\begin{document}


\begin{center}
{به نام خدا}
\end{center}

\hspace{8mm}

{\noindent \Large \textbf {
دانشگاه تهران، دانشکده مهندسی برق و کامپیوتر \\
تحلیل و طراحی الگوریتم‌ها \\
}}

{\noindent {
تمرین کتبی اول \\
موعد تحویل: شنبه ۴ اسفند ۹۷، ساعت ۹:۰۰ \\
طراح: نام طراح،
\texttt{johndoe@example.com}
}}

\begin{flushleft}
\nothing \\[-3.2cm]
\includegraphics[height=2.5cm]{../../../static/pics/ut-eng.png}
\end{flushleft}

%\maketitle
\rule{\textwidth}{1pt}


\begin{enumerate}
\item
سوال یک

\item
سوال دو
\begin{enumerate}
\item
بخش اول

\item
بخش دوم
\end{enumerate}

\item
سوال سه

\item
سوال چهار
\begin{itemize}
\item
فرض یک
\begin{itemize}
\item 
زیرفرض یک
\item
$f^-(u)=\sum_{e=(u,v)}f(e) = \sum_{e=(v,u)} f(e)=f^+(u)$

\end{itemize}  
 حال ثابت کنید:
 \begin{enumerate}
 \item
حکم یک
 
 \item
حکم دو
 
 \end{enumerate} 
 
\end{itemize}

\end{enumerate}

\end{document}
